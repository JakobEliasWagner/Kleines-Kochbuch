\section{Allergien und Unverträglichkeiten}\label{sec:allergien-und-unvertraglichkeiten}
Es sei darauf hinzuweisen, dass ihr auf jeden Fall schon der Freizeit den Allergien- und Unverträglichkeitenstatus aller Teilnehmenden abzufragen habt.
Frühzeitiges Abfragen erlaubt es euch bei ungewissheiten nochmals bei den Teilnehmenden, oder deren Eltern, anzurufen und erneut klar zu verstehen auf was geachtet werden muss.
Zu meist kann so abgeschätzt werden, ob ihr in der Küche diesen Dingen gerecht werden könnt.
Ist dies nicht möglich solltet ihr weiter kommunizieren und alternativen finden, damit die Person etwas zu essen bekommt.
Als Beispiel: Bei Zöliakie kann es sein, dass die Person unter keinen Umständen mit Mehl oder auch nur Mehlstaub in Kontakt kommen darf.
Dies bezieht sich hier auch auf Geräte und andere Dinge.
Es können also keine Oberflächen, keine Maschinen, keine weiteren Gegenstände benutzt werden, die bereits in der Vergangenheit mit Mehlstaub in Kontakt gekommen sind.
Dies ist eine Sache, die die Küche nicht gewährleisten kann und es muss daher unbedingt für eine Alternative gesorgt werden.
Wenn ihr euch unsicher seid, fragt unbedingt die Person/deren Eltern und erfahrene Köche und Köchinnen.
