\section{Allgemeine Tipps}\label{sec:allgemeine-tipps}

\begin{itemize}
    \item Wenn ihr für Menschen kocht, die bestimmten Regeln beim Essen folgen (Kosher, Halal, Vegetarisch, Vegan) müsst ihr unbedingt darauf achten Zutaten bei der Lagerung und beim Zubereiten nicht zu vermischen.
    \item Es kann sinnvoll sein für jedes Gericht oder Tage eine Person in der Gruppe zu definieren die besonders darauf achten soll, dass die Gerichte gelingen.
    Ohne diese Person kann es schnell mal passieren, dass man den Überblick über die Zeit verliert.
    \item Wer für kleine Gruppen kochen kann, kann theoretisch auch für große Kochen.
    Dennoch gibt es ein paar Dinge die anders sind:
    \begin{itemize}
        \item Arbeiten brauchen viel mehr Zeit.
        Kartoffeln schälen oder Zwiebeln und Gemüse schneiden können schnell Stunden dauern.
        Ihr solltet also unbedingt darauf achten, dass ihr genügend Zeit einplant und be verschiedenen Aufgaben eine lineare Skalierung der Arbeitszeit einplant.
        \item Würzen in großen Mengen ist schwierig.
        Wenn du also gut für kleinere Gruppen kochen kannst, trenne einfach eine kleinere Menge des Gerichts ab und schmecke es ab.
        Sobald ihr zufrieden seid damit, versucht die große Menge abzuschmecken.
        Dann könnt ihr abwechselnd die kleine und die große Menge probieren und so feststellen wie viel Gewürz noch im großen Topf fehlt.
        \item
    \end{itemize}
    \item Einen Blick für Ordnung zu halten ist auch wichtig, wenn es stressig wird.
    Wenn ihr immer die Dinge sofort aufräumt, die ihr nicht mehr braucht, bleibt die Küche sehr sauber.
    In einer gut organisierten und sauberen Küche kann man viel schneller und besser Kochen.
    Die so investierten Sekunden sind also keineswegs eine Verschwendung, sondern gehören zu effizientem Arbeiten dazu.
    \item Ab einer bestimmten größe skalieren \glqq normale\grqq{} Küchengeräte nicht mehr.
    Als Beispiel in Pfannen lassen sich Dinge nur bis zu einer bestimmten Menge gut anbraten.
    Danach hat entweder die Pfanne nicht mehr genügend power, oder die Sachen lassen sich einfach nicht mehr gut anbraten.
    Genau so verhält es sich mit ganz vielen anderen Dingen: Mixer, Pürierstab, Flotte-Lotte, usw.
    Die Lösung hierfür sind entweder Großgruppengeräte (Kipper im Beispiel der Pfanne), oder die Menge herunterzubrechen und alles in kleineren Mengen zu verarbeiten.
    \item Mengen beim Kochen für Großgruppen sind schwierig zu schätzen.
    Gerade wenn ihr unerfahren seid, solltet ihr unbedingt immer ausrechnen wie viel ihr von welcher Zutat braucht.
    \item Knoblauch und andere start riechende Zutaten sollten zu großen Teilen reduziert werden.
    Bei Rezepten die vom Kleinen hochgerechnet werden, sind die Mengen oftmals viel zu viel.
    Das ist in Ordnung, wenn ihr mit Freunden oder Familie esst, aber nicht für große Gruppen.
    Knoblauch schmeckt stark vor und skaliert auch nicht linear.
    \item Je größer die Gruppe ist, desto weniger müsst ihr für Einzelpersonen einrechnen.
    Durch regression zu mitte können ab 20 Personen 10\% und ab 50 Personen fast 20\% abgezogen werden.
    \item Achtet auf eine ausgewogene Ernährung.
    Zu viel fettige Ernährung freut zwar die Kinder, es liegt aber auch sehr schwer im Magen und kann zu Verstimmungen führen.
    \item Ihr als Küche seid zu einem großen Teil verantwortlich für die Stimmung in der Gruppe.
    Kocht also Dinge, mit denen ihr euch sicher seid (mindestens ein mal davor schon gekocht) und die auch dem Niveau der Gruppe entspricht.
    Eine scharfe Kürbissuppe mag zwar schmecken, ist aber auf der Kinderfreizeit unangebracht.
\end{itemize}