\section{Kalkulieren}\label{sec:kalkulieren}
Kalkulieren für größere Gruppen kann sehr schwierig sein.
Zu den Schwierigkeiten gehören Sachen wie die Zusammenstellung der Gruppe, Alter und auch die aktivität der Gruppe.
So muss für eine Ski-Freizeit anders kalkuliert werden als für ein Kinderzeltlager.
Jede Gruppe ist individuell zu betrachten und es ist unmöglich im vorhinein zu erahnen wie viel die Gruppe tatsächlich isst.
Auch kann sich die Menge die gekocht werden muss von Tag zu Tag unterscheiden.
Für gewöhnlich essen Gruppen an dem ersten Tag weniger als nach einer Eingewöhnungsphase.
Auch deswegen ist es sinnvoll bei längeren Freizeiten zuerst für die erste Hälfte der Woche einzukaufen.
Erst später für die restliche Woche einzukaufen verhindert zu viel wegschmeißen zu können.
Bei sehr kurzen Freizeiten ist dies aber nicht möglich und es muss einfach geschätzt werden.
Am einfachsten geht das mit Faustregeln.
Solche Faustregeln, mit unseren Erfahrungen, haben wir hier aufgeführt.


\begin{table}[H]
    \centering
    \caption{Kalkulationshilfe}
    \label{tab:kalkulationshilfe}
    \begin{tabular}{lll}
        \hline
        Zutat&Vorspeise&Hauptgericht\\\hline
        Reis&idk&mehr\\\hline
    \end{tabular}
\end{table}
