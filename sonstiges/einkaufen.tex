\section{Einkaufen}\label{sec:einkaufen}
Das Einkaufen beginnt schon bevor deine Großveranstaltung überhaupt schon begonnen hat.
Um gut einzukaufen, solltest du dich immer gut vorbereiten.
Dazu gehört:
\begin{itemize}
    \item Die Kalkulation deines Essens sollte vor dem Einkaufen abgeschlossen sein.
    Nur kleine Dinge sollten jetzt noch geändert oder anders eingekauft werden.
    Drucke die Kalkulation aus und nehm sie auf einem Klemmbrett mit.
    So können dann auch mehrere Personen gleichzeitig einkaufen und Dinge von der Liste streichen.
    Verwende auf solchen Listen Mengenangaben die jede*r verstehen kann (Falsch: \glqq 1 Dose Bohnen\grqq , Richtig: \qlqq 400g Dosenbohnen\grqq).
    \item Gute Organisation ist hier unglaublich wichtig.
    Hab einen gut organisierten Geldbeutel.
    Wenn du in einen Großmarkt möchtest, bemühe dich Frühzeitig um die Karte dafür.
    \item Erkundige dich im Vorhinein über den Kostenrahmen, mit dem für die Freizeit ausgekommen werden muss.
    \item Pack gute Kisten und vor allem wenn vorhanden Kühlkisten mit ein (Kühlakkus!).
    Anders kann es sonst schwierig werden das Eingekaufte überhaupt in das Auto zu bekommen.
    \item Kommuniziere vor dem Einkaufen nochmal mit der Hauptleitung oder dem Team der Freizeit.
    Es passiert immer wieder, dass zusätzliche Dinge eingekauft werden müssen.
\end{itemize}
Beim Einkaufen kann entweder im Großmarkt oder im Supermarkt eingekauft werden.
Aus meiner Erfahrung sollten im Großmarkt wirklich nur Dinge eingekauft werden, die in nicht Haushaltsüblichen Mengen gebraucht werden.
Dazu können Knödelbrot, Zwiebeln, Wurst, Käse, Joghurt, alles in Dosen und ähnliches gehören.
Wird im Großmarkt alles eingekauft, kann das schon schnell teurer werden, als wenn man dies einfach im Supermarkt kauft.
Nur in großen Gebinden werden die Preise in Großmärkten erheblich billiger als in Supermärkten.
Großmärkte haben zudem den Nachteil, dass es sehr schwierig ist überhaupt Dinge zu finden.
Großmärkte sind eher eingerichtet wie ein Lagerhaus und wenn man sich hier nicht auskennt, dauert das Einkaufen viel zu lang.
Mein Tipp ist also große Gebinde im Großmarkt und kleinere im Supermarkt zu kaufen.

Weiter sei darauf hinzuweisen, dass wenn für sehr viele Leute und viele Gerichte eingekauft wird, das Einkaufen sehr lange dauert.
Abhängig davon wie erfahren du bist, kann es durchaus Sinn machen am Vortag einzukaufen, um dafür ein Gefühl zu bekommen.
Ansonsten Plane genügend Zeit ein.

Achte beim Einkaufen auch eine hygienisch sinnvolle Reihenfolge einzuhalten.
Kaufe also Kühlgut nicht als Erstes ein.