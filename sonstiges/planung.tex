\section{Planung}\label{sec:planung}

Für die Planung von Essen solltet ihr auf die folgenden Dinge achten:
\begin{enumerate}
    \item Reichlich Gemüse, Kartoffeln und Obst
    \item Vielseitig
    \item Wenig Fett, wenig fettreiche Lebensmittel
    \item Würzig, aber nicht salzig
    \item Wenig Süßes
    \item Wenig tierisches Eiweiß
    \item Öfter mal Snacks (Vor allem Obst)
    \item Schmackhaft für alle (Schärfe, Produkte, die nicht allen schmecken (Koriander) und so weiter)
\end{enumerate}

\subsection{Vollwertigkeit}\label{subsec:vollwertigkeit}
Eigentlich kann zu fast jeder Mahlzeit Salt gegessen werden.
Salat ist gesund, erhöht die Abwechslung und steuert zu einem vollwertigen Essen bei.

\subsection{Abwechslungsreiche Zusammenstellung}\label{subsec:abwechslungsreiche-zusammenstellung}
Eintönigkeit vermeiden!
Abwechslungsreiche und schmackhafte Zusammenstellung der Gerichte unter Berücksichtigung von Verdaulichkeit, Sättigung und Vielseitigkeit.
Es kann auch hilfreich sein sich zu überlegen welche Kohlenhydrat-Quellen zu dem Essen gereicht werden können.
Sind das zum Beispiel Nudeln, Kartoffeln und Reis, kann es sinnvoll sein diese im Rotationsprinzip abzuwechseln.
Das, was es dann \glqq dazu \grqq{} gibt kann, dann außen herumgeplant werden.

\subsection{Arbeit und Zeitaufwand}\label{subsec:arbeits-und-zeitaufwand}
Alles was in der Großküche gemacht wird, dauert viel länger als Zuhause.
Es macht immer Sinn Zuhause Gerichte nochmal zu kochen und mit-zu-stoppen wie lange einzelne Arbeitsschritte dauern.
Dann können die Arbeitsschritte eigentlich linear mit der Menge skaliert werden.
Nudeln kochen an sich dauert zwar genauso lange wie Zuhause, nur Nudeln abgießen, in Schüsseln machen und so weiter dauert dann einfach viel länger.
Es sollte also immer ein Bewusstsein darüber geben wie lange es dauert bestimmte Arbeitsschritte auszuführen und auch ein Puffer vorhanden sein.
Weiter sollte auch immer darauf geachtet werden, welche Gerichte für wie viel mit wie vielen gekocht werden.
Einige Gerichte dauern deutlich kürzer als andere.
Gerade für unerfahrene Küchen sollte darauf geachtet werden, dass auf zu komplizierte Gerichte oder Zubereitungsformen (alles selber machen) verzichtet wird.
Es bringt nichts, wenn die Gerichte zwar auf dem Papier qualitativ hochwertig sind, die Ausführung dann aber so überfordert, dass das Endprodukt nicht gut ist.

Kleiner Tipp: Gerichte in der Küche müssen nicht in der Reihenfolge zubereitet werden, wie sie auf dem Speisezettel vorkommen.
Wenn vor einem Gericht noch Zeit ist, können schon Dinge vorbereitet werden, die ein späteres Gericht betreffen.
Also immer über die nächste Mahlzeit hinweg schauen und schon mal andere Dinge vorbereiten (Mise en Place) und auch sauber machen.
Je mehr Stress vermeiden kann, desto schöner ist die Zeit in der Küche.

\subsection{Kostenaufwand}\label{subsec:kostenaufwand}
Die Höhe der Kosten sollte immer darauf angepasst sein, wie Geld zu verfügung ist.
Gerade exotische Zutaten können sehr teuer werden und damit den kompletten Rahmen sprengen.
Achtet deshalb auf eine halbwegs regionale Auswahl von Gerichten.
Damit können schon sehr viele der großen Kostenpunkte vermieden werden.
Weiter ist es wichtig während des einkaufens ein ungefähres Gefühl davon zu behalten, an welchem Kostenstand die Planung gerade ist.
So kann auf der einen Seite mit einem Puffer geplant werden, oder auch kleine Schmankerl noch dazu gekauft werden, um das Essen aufzuwerten.

Kleiner Hinweis: Abhängig von der Planung kann sich der Kostenaufwand um mehrere Faktoren unterscheiden.
Soll sehr preiswert gekocht werden, können schon 3-5€ pro Person pro Tag reichen.

\subsection{Regelmäßige Mahlzeiten}\label{subsec:regelmaige-mahlzeiten}
Regelmäßigkeit und Pünktlichkeit der Mahlzeiten, Ruhe und eine freundliche Atmosphäre bei Tisch sind wichtige Zufriedenheit und Wohlbefinden aller Gruppenmitglieder.
Achtet also auch darauf den Essensaal gemütlich einzurichten und euch zu überlegen, wie ihr diese noch aufwerten könntet.
Dies kann zum Beispiel durch thematisch passende Dekoration, Blumen, Kerzen oder einfach nur durch auch mal draußen Essen umzusetzen sein.
Der Kreativität sind hier keine Grenzen zu setzen.