\section{Hygiene}\label{sec:hygiene}
Das wichtigste vorab: Ihr braucht eine Hygienebelehrung.
Diese Hygienebelehrung wird vom Gesundheitsamt gegeben und findet mehrmals die Woche statt.
Eine Kopie von dieser müsst ihr eurem Hauptverantwortlichen vor der Freizeit vorzulegen und dieser muss diese über die ganze Freizeit bei sich behalten.
Bei Kontrollen des Gesundheitsamts muss dieser diese dann vorlegen.
Hier nochmal die wichtigsten Dinge zusammen gefasst.
\subsection{Persönliche Hygiene}\label{subsec:personliche-hygiene}
Ringe und Armschmuck muss unbedingt abgelegt werden.
Die Haare zurückninden.
Hände gründlich waschen (auch nachdem man sich aus Versehen mal ins Gesicht gefasst hat).
Saubere Schürzen, Handtücher, Lappen und Schwämme verwenden.
Letztere sollten spätestens nach einem Tag ausgewechselt werden.
Wunden an den Händen müssen wasserdicht abgedeckt werden.
Hierzu solltet ihr immer Fingerlinge oder Handschuhe in eurem Hygieneset für die Küche haben.
Niemals in die Richtung der Speisen husten oder niesen.

\subsection{Hygiene am Arbeitsplatz}\label{subsec:hygiene-am-arbeitsplatz}
Arbeitsflächen immer sauber halten und vor der Arbeit abwischen.
Küchen in Jugendhäusern sind ganz und gar nicht sauber.
Ihr solltet bevor ihr anfangt zu arbeiten die ganze Küche anschauen und bei zu grober Verschmutzung der Gastgeber:in zurückmelden.
Lappen/Tücher nur ihrem entsprechenden Zweck gemäß verwenden.
Kühlschrank und Lagerstätten regelmäßig reinigen.
Arbeitsplatz immer sofort nach dem Kochen säubern.
Schneidebretter müssen eine glatte Oberfläche haben.
Reine und unreine Arbeitsschritte müssen räumlich oder zeitlich voneinander getrennt sein.
Nach unreinen Arbeitsschritten den gesamten Arbeitsbereich gründlich reinigen.


\subsection{Lebensmittelhygiene}\label{subsec:lebensmittelhygiene}
Obst und Gemüse vor der Verwendung gründlich waschen.
Sauberes Geschirr verwenden.
Auch hier ist es ratsam das Geschirr sogar noch vor der ersten Benutzung nochmals gründlich zu reinigen.
Die Kühlkette nicht unterbrechen (Kühlakkus mit zum Einkaufen nehmen).
Speisen nicht immer mit einem Löffel probieren.
Entweder immer einen neuen Löffel nehmen oder mit zwei Löffeln probieren.
Fleisch und andere Lebensmittel möglichst räumlich getrennt voneinander lagern (zwei Kühlschränke).
Weiter sollten alle Waren die beschädigt sind, entweder sofort verwendet werden oder weggeschmissen werden.
Alle Zutaten sollten vor der Verwendung nochmals auf ihre unversehrtheit überprüft werden.
Schimmelnde waren müssen immer weggeschmissen werden.
Es kann nicht einfach die schimmelig aussehende Stelle weggeschnitten werden.
Der Schimmel ist noch viel tiefer in den Waren und stark gesundheitsschädlich.
Lebensmittel niemals auf dem Boden lagern.
Rohes und gegartes getrennt voneinander lagern.
