\section{Umweltbewusste Ernährung}\label{sec:umweltbewusste-ernahrung}
Wenn ihr für euch selber kocht, tragt ihr die Verantwortung für die Auswirkungen eurer Ernährung.
Kocht ihr für große Gruppen, übernehmt ihr die Verantwortung von der gesamten Gruppe.
Dieser Verantwortung solltet ihr gerecht werden.


\begin{itemize}
    \item \textbf{Regional}: Regional Kochen trägt leider nicht so stark zu einer guten Umweltperformance bei.
    Ihr solltet aber trotzdem darauf achten, woher eure Zutaten kommen.
    Achtet darauf, keine Zutaten zu verwenden die erst eingeflogen oder geschifft werden müssen (exotische Früchte, Avocado, usw.).
    \item \textbf{Saisonal}: Achtet dadrauf der Saison gerechte Lebensmittel zu verwenden.
    \item \textbf{Fleisch}: Fleisch trägt ungeheuer groß und stark zur umweltperformance eines Gerichts bei.
    Verzichtet so weit das möglich ist auf Fleisch zu verzichten.
    Auch der Mythos, dass das Substituieren von Käse und Milch in vegetarischen gerichten zu einer schlechteren Performance führt, gegenüber dem Gericht mit Fleisch, ist falsch.
    Weiter sei in diesem Zuge aber zu sagen, dass es hier auch nicht falsch ist Käse und Milchprodukte zu reduzieren.
    Auch diese sind nicht gut für die Umwelt.
\end{itemize}
