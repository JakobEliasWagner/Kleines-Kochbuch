\section{Hauptgerichte}\label{sec:hauptgerichte}
\newpage

\subsection{Nudeln mit Tomatensauce}\label{subsec:nudeln-mit-tomatensauce}
\begin{tcolorbox}
    [
    blanker,
    width=0.64\textwidth,enlarge left by=0.36\textwidth,
    before skip=6pt,
    breakable,
    overlay unbroken and first={%
        \node[inner sep=0pt,outer sep=0pt,text width=0.33\textwidth,
            align=none,
            below right]
        at ([xshift=-0.36\textwidth]frame.north west)
            {%% left
            Rezept von Jakob\\
            Rezept für 10\\
            \begin{flushright}
                \noindent\makebox[\linewidth]{\rule{\linewidth}{0.4pt}}
                \textbf{Tomatensauce}\\
                Zwiebeln\\
                Sellerie\\
                Karotten\\
                Tomatenmark\\
                Tomaten aus der Dose\\
                \textbf{Nudeln}\\
                Nudeln
            \end{flushright}
        };}]
%% right
    \begin{figure}[H]
        \begin{center}
%\includegraphics[width=\wdimg]{img/•}
            \caption{Bild}
        \end{center}
    \end{figure}
%%%%%%%%%% Rezept Anleitung %%%%%%
    Für die Tomatensauce die Zwiebeln, Karotten und Sellerie entweder fein schneiden oder reiben.
    Alle drei dann zusammen bei mittlerer Hitze anschwitzen.
    Die Zwiebeln sollten vollständig ihren scharfen Geruch verloren haben.\\
    Anschließend etwas Tomatenmark hinzugeben und dieses mit anschwitzen.
    Hierbei vorsichtiger sein als daheim: Gerade in großen Töpfen und mit Herden, die unbekannt sind, kann dieses schnell anbrennen.
    Anschließend mit Wasser oder Brühe abschrecken.
    Danach den Boden des Topfes frei Kratzen.
    Die Sauce heiß werden lassen und anschließend die Hitze auf niedrig vermindern.
    Während des Kochens wiederholt umrühren und ein Gefühl dafür entwickeln ob Sachen in den Töpfen anbrennen, oder eben nicht.\\

    Noch als kleiner Hinweis: Wird mit Rotwein gekocht (oftmals auch nicht vegan oder vegetarisch), verkocht der darin enthaltene Alkohol konträr der Annahme nicht vollständig.
    Gerade wenn für Kinder gekocht wird, kann es folglich nicht angemessen sein Wein zu benutzen.

    Nudeln für viele Personen kochen ist nicht ganz einfach.
    Für die Nudeln ausreichend Wasser zum Kochen bringen (Normalerweise sagt man ungefähr 100g Nudeln auf 1l Wasser), was aber in der Realität überhaupt nicht umsetzbar ist.
    Kocht die Nudeln auch nicht nach der Anleitung auf der Packung, sondern probiert immer wieder.
    Nudeln, besonders wenn sie warm sind, ziehen nach und wenn man sie  \glqq fertig \grqq{} kocht, werden sie im Anschluss noch weiter kochen und super weich.
    Ich mache es meistens so, dass ich die Nudeln noch recht al dente aus dem Wasser nehme und sie dann im Topf oder den Gefäßen mit denen man sie raus stellt fertig koche.
\end{tcolorbox}
\newpage

\subsection{Semmelknödel}\label{subsec:semmelknodel}
\begin{tcolorbox}
    [
    blanker,
    width=0.64\textwidth,enlarge left by=0.36\textwidth,
    before skip=6pt,
    breakable,
    overlay unbroken and first={%
        \node[inner sep=0pt,outer sep=0pt,text width=0.33\textwidth,
            align=none,
            below right]
        at ([xshift=-0.36\textwidth]frame.north west)
            {%% left
            Rezept von Jakob\\
            Rezept für 10\\
            \begin{flushright}
                \noindent\makebox[\linewidth]{\rule{\linewidth}{0.4pt}}
                \textbf{Sub Zutat}\\
                Menge Zutat
            \end{flushright}
        };}]
%% right
    \begin{figure}[H]
        \begin{center}
%\includegraphics[width=\wdimg]{img/•}
            \caption{Bild}
        \end{center}
    \end{figure}
%%%%%%%%%% Rezept Anleitung %%%%%%

\end{tcolorbox}
\newpage

\subsection{Curry mit Reis}\label{subsec:curry-mit-reis}
\begin{tcolorbox}
    [
    blanker,
    width=0.64\textwidth,enlarge left by=0.36\textwidth,
    before skip=6pt,
    breakable,
    overlay unbroken and first={%
        \node[inner sep=0pt,outer sep=0pt,text width=0.33\textwidth,
            align=none,
            below right]
        at ([xshift=-0.36\textwidth]frame.north west)
            {%% left
            Rezept von Jakob\\
            Rezept für 10\\
            \begin{flushright}
                \noindent\makebox[\linewidth]{\rule{\linewidth}{0.4pt}}
                \textbf{Sub Zutat}\\
                Menge Zutat
            \end{flushright}
        };}]
%% right
    \begin{figure}[H]
        \begin{center}
%\includegraphics[width=\wdimg]{img/•}
            \caption{Bild}
        \end{center}
    \end{figure}
%%%%%%%%%% Rezept Anleitung %%%%%%

\end{tcolorbox}
\newpage
