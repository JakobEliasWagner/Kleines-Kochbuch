\section{Snacks}\label{sec:snacks}

\newpage

\subsection{Stockbrot}\label{subsec:stockbrot}
\begin{tcolorbox}
    [
    blanker,
    width=0.64\textwidth,enlarge left by=0.36\textwidth,
    before skip=6pt,
    breakable,
    overlay unbroken and first={%
        \node[inner sep=0pt,outer sep=0pt,text width=0.33\textwidth,
            align=none,
            below right]
        at ([xshift=-0.36\textwidth]frame.north west)
            {%% left
            Rezept von Jakob\\
            Rezept für 10\\
            \begin{flushright}
                \noindent\makebox[\linewidth]{\rule{\linewidth}{0.4pt}}
                \textbf{Sub Zutat}\\
                Menge Zutat
            \end{flushright}
        };}]
%% right
    \begin{figure}[H]
        \begin{center}
%\includegraphics[width=\wdimg]{img/•}
            \caption{Bild}
        \end{center}
    \end{figure}
%%%%%%%%%% Rezept Anleitung %%%%%%

\end{tcolorbox}
\newpage

\subsection{Flammkuchen}\label{subsec:flammkuchen}
\begin{tcolorbox}
    [
    blanker,
    width=0.64\textwidth,enlarge left by=0.36\textwidth,
    before skip=6pt,
    breakable,
    overlay unbroken and first={%
        \node[inner sep=0pt,outer sep=0pt,text width=0.33\textwidth,
            align=none,
            below right]
        at ([xshift=-0.36\textwidth]frame.north west)
            {%% left
            Rezept von Jakob\\
            Rezept für 10\\
            \begin{flushright}
                \noindent\makebox[\linewidth]{\rule{\linewidth}{0.4pt}}
                \textbf{Sub Zutat}\\
                Menge Zutat
            \end{flushright}
        };}]
%% right
    \begin{figure}[H]
        \begin{center}
%\includegraphics[width=\wdimg]{img/•}
            \caption{Bild}
        \end{center}
    \end{figure}
%%%%%%%%%% Rezept Anleitung %%%%%%

\end{tcolorbox}
\newpage
